% Section 3.- La Población y la Muestra
\section{La Población y la Muestra}
    %Subsection 3.1.- Población
    \subsection{Población}
        Se entiende por población o universo a la totalidad de individuos o elementos 
        en los cuales puede presentarse determinada característica susceptible de ser 
        estudiada. Generalmente, ese conjunto viene definido por comprensión, es decir, 
        citando la propiedad que caracteriza a sus elementos. Los datos individuales que una
        población se llaman unidades elementales u observaciones. A continuación, se presentamos
        algunos ejemplos: 
        \begin{itemize}
            \item Población de ventas anuales de los supermercados de Lima Metropolitana.
            \item Población de todos los posibles resultados cara o sello que se obtenga al arrojar una moneda un número indefinido de veces.
            \item Población de puntajes de rendimiento en la lectura de todos los alumnos del nivel primario en un sistema solar.
        \end{itemize}
        La población puede ser finita o infinita, dependiendo del número de elementos que la conforman.
        \begin{itemize}
            \item \textbf{Población finita.-} Es aquélla que tiene un número determinado de elementos .
            \item \textbf{Población infinita.-} Es aquélla que tiene un número infinito de elementos. En la práctica, una población finita con un número grande de elementos se considera una población infinita.
        \end{itemize}
    %Subsection 3.2.- Muestra
    \subsection{Muestra}
        Al conjunto de medidas o conteos que se obtienen de alguna población
        con el propósito de obtener información acerca de ella se le da el nombre de muestra. 
        Se suelen tomar muestras cuando es difícil o costosa la observación de todos los
        elementos de la población estadística. Al número de elementos de la muestra 
        se le llama tamaño de la muestra .