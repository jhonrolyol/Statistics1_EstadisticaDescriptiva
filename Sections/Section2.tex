% Section 2.- Ramas de la estadística
\section{Ramas de la estadística}
    Se han planteado multiples definiciones de la estadística, algunas 
    caracterizando la estadística como ciencia, y otras como metodología. Con fines
    estrictamente instructivos daremos la siguiente definición. 
        % Definición
        \begin{defi}
            Estadística es la ciencia que nos proporciona un conjunto de métodos
            y procedimientos para recolección, clasificación (organización), análisis
            e interpretación de datos en forma adecuada para tomar decisiones cuando 
            prevalecen condiciones de incertidumbre.
        \end{defi}
    De acuerdo a esta definición podemos clasificar la estadística en :
        \begin{enumerate}
            \item Estadística Descriptiva
            \item Estadística Inferencial
        \end{enumerate}
    % Subsection 2.1.- Estadística Descriptiva
    \subsection{Estadística Descriptiva}
        Es la parte de la estadística que se encarga de la recolección, clasificación, 
        presentación, descripción y simplificación de los datos. En otras palabras, podemos 
        expresar que un estudio estadístico se considera ``descriptivo'' cuando solamente 
        se pretende analizar y describir los datos. Particularmente, podemos resumir la 
        Estadística Descriptiva en el siguiente diagrama. 
        % Diagrama
        \begin{center}
            \mbox{Recolección de Datos} $\rightarrow$ \mbox{Crítica de Datos} $\rightarrow$ \mbox{Presentación (Tablas, Gráficas)}$\rightarrow$ \mbox{Análisis Descriptivo}
        \end{center}
    % Subsection 2.2.- Estadística Inferencial
    \subsection{Estadística Inferencial}
        Es lo que nos proporciona la teoría necesaria para inferir o estimar las leyes de una población partiendo de los resultados
        o conclusiones del análisis de una muestra. Es decir, podemnos considerar que un 
        estudio estadístico es inferencial cuando se pretende inferir conclusiones que atañen a una 
        población de donde procede la muestra y como estas conclusiones 
        nunca pueden ser absolutamente ciertas, ellas estarán ligadas a cierto grado 
        de incertidumbre o probabilidad.


